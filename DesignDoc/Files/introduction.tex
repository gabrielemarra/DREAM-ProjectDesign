\subsection{Purpose}
\begin{flushleft}

\end{flushleft}

\subsection{Scope}
\subsection{Definitions, Acronyms, Abbreviations}

\hl{\textbf{NOTE: this is copied from RASD}}
\begin{center}
\renewcommand{\arraystretch}{1.25}
\begin{tabular}{l >{\raggedright\arraybackslash}p{12cm} } \hline
    \textbf{Term} & \textbf{Definition}\\ 
    DREAM & The system described in this document; Data-dRiven PrEdictive FArMing\\
    User & Farmer, agronomist, or policy user; anyone who uses the system.\\
	Policy Maker & Member of the Telangana government who deploys and manages different agriculture-related policies. \\
	Agronomist & Professional who specializes in agriculture sciences. \\
    Farmer & A user who uses DREAM to help manage data relating to their farms and fields.\\
    Field & One enclosed area that corresponds to one crop. Many fields can make up a farm. The locations of the various fields do not need to be co-located.\\
    Farm & A set of one or many fields that are managed by one farmer.\\
    Production yields & The amount of crop harvested compared to the amount of crop planted. Measured comparatively by percentage or numerically by weight.\\
    Flag & A marker on a farmer that signals the system to increase the priority for the farmer to get visited by an agronomist.\\
    TSDPS & Telangana State Development Planning Society which manages the automated weather stations around the state. \\
    World & A graphical representation of an instance of the Alloy model.\\
    UML & Unified Modeling Language\\
    MTTF & Mean Time To Failure\\
    MTTR & Mean Time To Recovery\\
    \hline
\end{tabular}
\end{center}



\subsection{Revision History}
\begin{flushleft}
\renewcommand{\arraystretch}{1.25}
\begin{tabular}{|c| l|>{\raggedright\arraybackslash}p{12cm} |} \hline
    \textbf{Revision} & \textbf{Date} & \textbf{Description}\\ \hline 
    1.0 & 09 January 2022 & Initial Release.\\
    \hline
\end{tabular}
\end{flushleft}

\subsection{Reference Documents}
\subsection{Document Structure}
The document is structured with the following sections:
\begin{itemize}
\item \textbf{Introduction}:
\item \textbf{Architectural Design}:
\item \textbf{User Interface Design}:
\item \textbf{Requirements Traceability}:
\item \textbf{Implementation, Integration, and Test Plan}:
\item \textbf{Effort Spent}: For the purposes of the project assignment, this section itemizes the time each participate allotted to different phases of the project. 
\item \textbf{References}:
\end{itemize}




