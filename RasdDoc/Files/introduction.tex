\newcounter{goals_counter}
\setcounter{goals_counter}{1}
\subsection{The Problem}
\begin{flushleft}
%%%%%%%%%%%%%%%%%%%%%%%%%%%%%%%%%%%
%%%%%%%%%%% THE PROBLEM %%%%%%%%%%%
%%%%%%%%%%%%%%%%%%%%%%%%%%%%%%%%%%%
The world's food supply chain is threatened by climate change and an unsustainable rate of rising population. Since the Telangana region largely participates in the world's food economy, the Telangana government is interested in pursuing initiatives aimed at mitigating the effect of these issues. In this case, the Telangana government wants to reform the way they build their policies around food production. Their goal is to utilize digital public goods and community-centric approaches in order to build resiliency against these dynamic challenges by building policies that are more agile and data-driven. 

%%%%%%%%%%%%%%%%%%%%%%%%%%%%
%%%%%%%%% PURPOSE %%%%%%%%%%
%%%%%%%%%%%%%%%%%%%%%%%%%%%%
\subsection{Purpose}
\subsubsection{Purpose of the Document}
The purpose of this document is to provide a comprehensive description of the requirements and specifications of the project. This includes clearly defining the problem, then specifying the expectations of the proposed solution. In order to provide a comprehensive analysis of the requirements and specifications of the DREAM product, the product will be described in terms of the necessary functions required and a look at how the product will serve its specific users. Then, the requirements will be detailed including the interface requirements, functional requirements, and the performance requirements. The requirements defined will be subsequently analyzed using Alloy. 

\subsubsection{Purpose of the Project}
The purpose of this effort is to provide a solution to address the problem previously outlined. 
%%%%%%%%%%%%%%%%%%%%%%%%%%%%%%%%%%%
%%%%%% THE PROPOSED SOLUTION %%%%%%
%%%%%%%%%%%%%%%%%%%%%%%%%%%%%%%%%%%
This initiative, named {\bf D}ata-d{\bf R}iven Pr{\bf E}dictive F{\bf A}r{\bf M}ing, or DREAM, intends to provide a solution that focuses on serving the needs of three stakeholders: farmers, agronomists, and policy makers. 
\smallskip \\

%%%%%%%%%%%%%%%%%%%%%%%%%%%%%%%%%%%
%%%%%%%%% GENERAL DIAGRAM %%%%%%%%%
%%%%%%%%%%%%%%%%%%%%%%%%%%%%%%%%%%%
\begin{figure}
\centering
\includegraphics[scale=0.6]{../images_diagrams/stakeholders_in_broader_gov_objective.png}
\caption{The main objective of this initiative may involve many stakeholders but we will focus on farmers, agronomists, and policy holders.}\label{fig:addOne{figure_counter}}
\end{figure}


The general aim is to take advantage of the present-day abundance of data and potential for platforms to streamline communication. By introducing data analysis, farmers can have access to automated solutions, therefore tackling their issues with agility and confidence. Additionally, by offering a means of communication, farmers can leverage their own community as a resource for information and support. Regarding policy makers and agronomists, the product will serve as a tool to aid their respective work-flows. 
\end{flushleft}

%%%%%%%%%%%%%%%%%%%%%%%%%%%%%%%%%%%
%%%%%%%%%%%% THE GOALS %%%%%%%%%%%%
%%%%%%%%%%%%%%%%%%%%%%%%%%%%%%%%%%%
\subsubsection{Goals}

DREAM will serve as a central tool for policy makers to access data about the region. For farmers, DREAM will serve as a tool to help them organize their own production data as well as provide an interface to seek for support from their community. Finally, agronomists will use DREAM as a tool to manage their work such as planning field visits or verifying the data provided by farmers. The following table lists an aggregate collection of the distinct goals that serve the three main stakeholders in the scope of this system.

%%%% GOALS TABLE %%%%
\renewcommand{\arraystretch}{1.25}
\begin{table}
\centering
\caption{\label{tab:addOne{table_counter}}Your caption.}
\begin{tabular}{|c| >{\raggedright\arraybackslash}p{12cm}|} \hline
    \textbf{ID} & \textbf{Goals}\\
    \hline
    G\addOne{goals_counter}  & Farmers can visualize relevant data and suggestions based on their location and type of production.\\ 
    \hline
    G\addOne{goals_counter}  & Agronomists and farmers can view weather forecast data.\\ 
    \hline
    G\addOne{goals_counter}  & Farmers can interact with others farmers and agronomists by requesting for help and suggestions.\\
    \hline
    G\addOne{goals_counter}  & Farmers can create discussion forums with other farmers.\\
    \hline
    G\addOne{goals_counter}  & Agronomists can supervise a sub-area inside the region. \\
    \hline
    G\addOne{goals_counter}  & Agronomists can visualize the performance of the farmers in their sub-area.\\ %view the ranking of farmers’ performance in their specific area.
    \hline
    G\addOne{goals_counter}  & Agronomists can visualize and update a daily plan to visit farms in their area.\\
    \hline
    G\addOne{goals_counter}  & Agronomists can specify the deviations from their daily plan and confirm the execution of their daily plan at the end of each day.\\
    \hline
    G\addOne{goals_counter}  & Telengana’s policy makers can view the performance of the farmers and the ranking of the farmers.\\
    \hline
    G\addOne{goals_counter} & Telengana’s policy makers can determine if support from agronomists and well-performing farmers produces significant results.\\
    \hline
\end{tabular}
\end{table}


%%%%%%%%%%%%%%%%%%%%%%%%%%%%%%%%%%%
%%%%%%%%%%%%% PHENOMENA %%%%%%%%%%%
%%%%%%%%%%%%%%%%%%%%%%%%%%%%%%%%%%%
\subsection{Scope}

\begin{flushleft} 
Considering the three users in the scope of this initiative, the design of the system must first consider the following phenomena in the context in which the system will operate.  

%%%% PHENOMENA TABLE %%%%
\subsubsection{World Phenomena}


\newcounter{phenomena_counter}
\setcounter{phenomena_counter}{1}
\begin{table}[hbt!]
\centering
\caption{\label{tab:worldphenomena} Phenomena related to the world.}
\renewcommand{\arraystretch}{1.25}
\begin{tabular}{|l|>{\raggedright\arraybackslash}m{12cm}|} \hline
    \textbf{World Phenomena} & \textbf{Description}\\\hline
	WP\addOne{phenomena_counter} & An agronomist visits a farm\\\hline
	WP\addOne{phenomena_counter} & Farmer has an issue with the farm\\\hline
\end{tabular}

\end{table}

\smallskip
\subsubsection{Shared Phenomena}
\newcounter{shared_counter}
\setcounter{shared_counter}{1}


\begin{table}[hbt!]
\centering
\small
\renewcommand{\arraystretch}{1.25}

\caption{\label{tab:sharedphenomena} Shared Phenomena.}
\begin{tabular}{|m{2cm}|m{10cm}|m{2cm}|}

\hline
\textbf{Shared \newline Phenomena}  & \textbf{Description} & \textbf{Controlled By} \\ \hline

SP\addOne{shared_counter} & An agronomist confirms a plan and send all the data about the visits he performed& W \\ \hline
SP\addOne{shared_counter} & A farmer sends a message to an agronomist &  W\\ \hline
SP\addOne{shared_counter} & A farmer creates a forum discussion  & W  \\ \hline
SP\addOne{shared_counter} & An agronomist responds to a farmer help or suggestion request  & W \\ \hline 
SP\addOne{shared_counter} & A user inspects data  & W \\ \hline
SP\addOne{shared_counter} & Policy maker flags poor performing and well performing farmers & W \\ \hline
\end{tabular}


\end{table}
\smallskip
\subsubsection{Machine Phenomena}




\newcounter{machine_phenomena}
\setcounter{machine_phenomena}{1}

\begin{table}[hp!]
\centering
\caption{\label{tab:addOne{table_counter}} Caption hello why why whyyyyy!}

\renewcommand{\arraystretch}{1.25}
\begin{tabular}{|l|>{\raggedright\arraybackslash}m{12cm}|} \hline
    \textbf{Machine Phenomena} & \textbf{Description}\\\hline
	MP\addOne{machine_phenomena} & An agronomist visits a farm\\\hline
	MP\addOne{machine_phenomena} & Farmer has an issue with the farm\\\hline
	MP\addOne{machine_phenomena} & Data analysis is performed \\ \hline
	MP\addOne{machine_phenomena} & Statistics are created based on data analyzed\\ \hline
	MP\addOne{machine_phenomena} & The system computes the best path connecting all farmers an agronomist has to visit \\ \hline
	MP\addOne{machine_phenomena} & The system recommend farmers to be visited by an agronomist\\ \hline

\end{tabular}
\end{table}
\smallskip
\end{flushleft}

\subsection{Definitions, Acronyms, Abbreviations}

%%%% DEFINITIONS TABLE %%%%

\begin{center}
\renewcommand{\arraystretch}{1.25}
\begin{tabular}{l >{\raggedright\arraybackslash}p{12cm} } \hline
    \textbf{Term} & \textbf{Definition}\\ 
    DREAM & The system described in this document; Data-dRiven PrEdictive FArMing\\
    User & Farmer, agronomist, or policy user; anyone who uses the system.\\
	Policy Maker & Member of the Telangana government who deploys and manages different agriculture-related policies. \\
	Agronomist & Professional who specializes in agriculture sciences. \\
    Farmer & A user who uses DREAM to help manage data relating to their farms and fields.\\
    Field & One enclosed area that corresponds to one crop. Many fields can make up a farm. The locations of the various fields do not need to be co-located.\\
    Farm & A set of one or many fields that are managed by one farmer.\\
    Production yields & The amount of crop harvested compared to the amount of crop planted. Measured comparatively by percentage or numerically by weight.\\
    Flag & A marker on a farmer that signals the system to increase the priority for the farmer to get visited by an agronomist.\\
    TSDPS & Telangana State Development Planning Society which manages the automated weather stations around the state. \\
    \hline
\end{tabular}
\end{center}

\subsection{Revision History}
\subsection{Reference Documents}
\subsection{Document Structure}
