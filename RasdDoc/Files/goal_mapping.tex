\newcommand\goal[1]{\item[[ G#1]]}
\newcommand\dom[1]{\item[[ D#1]]}
\newcommand\req[1]{\item[[ R#1]]}

Requirements must ensure satisfaction of the goal given the context of the domain assumption.


\begin{itemize}
\goal{1} \textbf{Farmers can visualize relevant data and suggestions based on their location and type of production.}

\begin{itemize}
\dom{1}  Users must have a device connected to internet.
\dom{2} To access to the system the user must have valid credentials.
\dom{3} The data about weather forecast, the farmers and their production, the sensors, the agronomist are correct, complete and stored by the application
\dom{4} The user has granted permission for GPS, notifications and disk usage
\dom{5} Farmers have an existing system to quantify, track, and organize their production yields
\dom{6} Users can successfully operate an interactive application

\req{1} The system must allow the farmer to set the production types of their fields.
\req{2} The system must allow the farmer to set the position of their fields manually.
\req{3} The system must allow the farmer to set the position of their fields through their devices' GPS.
\req{4} The system must keep track of the data about farmers
\req{5} The system must provide an interface to visualize data
\req{6} The system must be able to analyze data and show statistics
\req{7} The system must enable farmers to modify their production type.
\req{8} The system must enable farmers to report issues they may face.
\req{9} The system must allow the farmer to report production data at a frequency chosen by the farmer.
\req{10} The system must retrieve the weather forecast data from the data that the Telangana government collects.
\end{itemize}

\goal{2}  \textbf{Agronomists and farmers can view weather forecast data.}

\begin{itemize}
\dom{1}  Users must have a device connected to internet.
\dom{2} To access to the system the user must have valid credentials.
\dom{3} The data about weather forecast, the farmers and their production, the sensors, the agronomist are correct, complete and stored by the application
\dom{4} The user has granted permission for GPS, notifications and disk usage
\dom{6} Users can successfully operate an interactive application
\dom{7} Business competition will not influence the farmers' willingness to help
\dom{8} Farmers are willing to ask for help from other farmers and/or agronomists
\dom{9} Farmers have industry knowledge about fertilizers, crops, etc 
\dom{10} Farmers are willing to interact with other farmers 
\dom{12} Agronomists are assigned an area by their superiors
\dom{16} Farmers and Agronomists are not interested in meteorological changes that occur in less than 5 minutes

\req{2} The system must allow the farmer to set the position of their fields manually.
\req{3} The system must allow the farmer to set the position of their fields through their devices' GPS.
\req{4} The system must keep track of the data about farmers
\req{5} The system must provide an interface to visualize data
\req{10} The system must retrieve the weather forecast data from the data that the Telangana government collects.
\req{11} The system must show updated weather forecast data at most 5 minutes from which the data has been published by the Telangana government.
\req{12} The system must provide weather data that forecasts at least 3 days ahead
\req{13} The system must allow agronomists to access weather forecast data specific to their responsible area
\req{14} The system must allow farmers to access weather forecast data based on their GPS location [or from the location of thier farm on record]

\end{itemize}

\goal{3} \textbf{Farmers can interact with others farmers and agronomists by requesting for help and suggestions.}

\begin{itemize}
\dom{1}  Users must have a device connected to internet.
\dom{2} To access to the system the user must have valid credentials.
\dom{4} The user has granted permission for GPS, notifications and disk usage
\dom{5} Farmers have an existing system to quantify, track, and organize their production yields
\dom{6} Users can successfully operate an interactive application
\dom{7} Business competition will not influence the farmers' willingness to help
\dom{8} Farmers are willing to ask for help from other farmers and/or agronomists
\dom{9} Farmers have industry knowledge about fertilizers, crops, etc 
\dom{10} Farmers are willing to interact with other farmers 
\dom{11} Farmers can recognize issues and production abnormalities
\dom{12} Agronomists are assigned an area by their superiors
\dom{14} Agronomists are experts in their field
\dom{17} Agronomists are effective in determine performance based on various data points

\req{5} The system must provide an interface to visualize data
\req{6} The system must be able to analyze data and show statistics
\req{8} The system must enable farmers to report issues they may face.
\req{9} The system must allow the farmer to report production data at a frequency chosen by the farmer.
\req{10} The system must retrieve the weather forecast data from the data that the Telangana government collects.
\req{15} The system must provide an interface for farmers to request help and suggestions from other farmers
\req{16} The system must provide an interface for farmers to receive help requests and receive suggestions sent to them from other farmers
\req{17} The system must provide an interface for farmers to provide suggestions to other farmers
\req{18} The system must provide an interface for farmers to respond to help requests sent to them from other farmers
\req{19} The system must provide an interface for farmers to request help and suggestions from other agronomists
\req{20} The system must provide an interface for agronomists to recieve help requests sent to them from other farmers
\req{21} The system must provide an interface for agronomists to respond to help requests sent to them from other farmers 
\req{22} The system must provide an interface for agronomists to provide suggestions to other farmers
\end{itemize}

\goal{4} \textbf{Farmers can create discussion forums with other farmers}
\begin{itemize}
\dom{1}  Users must have a device connected to internet.
\dom{2} To access to the system the user must have valid credentials.
\dom{4} The user has granted permission for GPS, notifications and disk usage
\dom{5} Farmers have an existing system to quantify, track, and organize their production yields
\dom{6} Users can successfully operate an interactive application
\dom{7} Business competition will not influence the farmers' willingness to help
\dom{8} Farmers are willing to ask for help from other farmers and/or agronomists
\dom{9} Farmers have industry knowledge about fertilizers, crops, etc 
\dom{10} Farmers are willing to interact with other farmers 
\dom{11} Farmers can recognize issues and production abnormalities


\req{5} The system must provide an interface to visualize data
\req{8} The system must enable farmers to report issues they may face.
\req{23}  The system must provide a forum interface
\req{24}  The system must allow the farmer to create discussion forums.
\req{25}  The system must allow the farmer to create discussion forums.
\req{26}  The system must allow farmers to post replies in the discussion forum
\req{27}  The system must keep track of all the forum discussion
\end{itemize}

\goal{5} \textbf{Agronomists can supervise a sub-area inside the region}
\begin{itemize}
\dom{1}  Users must have a device connected to internet.
\dom{2} To access to the system the user must have valid credentials.
\dom{3} The data about weather forecast, the farmers and their production, the sensors, the agronomist are correct, complete and stored by the application
\dom{4} The user has granted permission for GPS, notifications and disk usage
\dom{5} Farmers have an existing system to quantify, track, and organize their production yields
\dom{6} Users can successfully operate an interactive application
\dom{12} Agronomists are assigned an area by their superiors
\dom{13} Agronomists can effectively manage an area assigned to them (ie, the agronomist is not overworked)

\req{5} The system must provide an interface to visualize data
\req{6} The system must be able to analyze data and show statistics
\req{9} The system must allow the farmer to report production data at a frequency chosen by the farmer.
\req{10} The system must retrieve the weather forecast data from the data that the Telangana government collects.
\req{13} The system must allow agronomists to access weather forecast data specific to their responsible area
\req{28} The system must allow agronomists to specify the geographic area in which they are responsible for.
\req{29} The system must allow agronomists to specify the geographic area in which they are responsible for.
\end{itemize}

\goal{6} \textbf{Agronomists can view the ranking of farmers’ performance in their specific area.}
\begin{itemize}
\dom{1}  Users must have a device connected to internet.
\dom{2} To access to the system the user must have valid credentials.
\dom{3} The data about weather forecast, the farmers and their production, the sensors, the agronomist are correct, complete and stored by the application
\dom{4} The user has granted permission for GPS, notifications and disk usage
\dom{5} Farmers have an existing system to quantify, track, and organize their production yields
\dom{6} Users can successfully operate an interactive application
\dom{12} Agronomists are assigned an area by their superiors
\dom{13} Agronomists can effectively manage an area assigned to them (ie, the agronomist is not overworked)
\dom{14} Agronomists are experts in their field
\dom{15} Agronomists will be effective in addressing issues farmers face


\req{5} The system must provide an interface to visualize data
\req{6} The system must be able to analyze data and show statistics
\req{8} The system must enable farmers to report issues they may face.
\req{9} The system must allow the farmer to report production data at a frequency chosen by the farmer.
\req{10} The system must retrieve the weather forecast data from the data that the Telangana government collects.
\req{13} The system must allow agronomists to access weather forecast data specific to their responsible area
\req{28} The system must allow agronomists to specify the geographic area in which they are responsible for.
\req{29} The system must allow agronomists to specify the geographic area in which they are responsible for.
\req{30}  The system must allow agronomist to view the list of all farmers in their area.
\req{31}  The system must provide an evaluation of farmers such that the exaluation reflects the quality and quantity of their crop production
\req{32}  The system must enable agronomists to access farmer evaluations from their specific area
\req{33}  The system updates farmers' evaluation when new data is available (ie, new farmer event entries or after an agronomist visit, etc)
\end{itemize}

\goal{7} \textbf{Agronomists can visualize and update a daily plan to visit farms in their area}
\begin{itemize}
\dom{1}  Users must have a device connected to internet.
\dom{2} To access to the system the user must have valid credentials.
\dom{3} The data about weather forecast, the farmers and their production, the sensors, the agronomist are correct, complete and stored by the application
\dom{4} The user has granted permission for GPS, notifications and disk usage
\dom{6} Users can successfully operate an interactive application
\dom{12} Agronomists are assigned an area by their superiors
\dom{13} Agronomists can effectively manage an area assigned to them (ie, the agronomist is not overworked)


\req{5} The system must provide an interface to visualize data
\req{6} The system must be able to analyze data and show statistics
\req{9} The system must allow the farmer to report production data at a frequency chosen by the farmer.
\req{13} The system must allow agronomists to access weather forecast data specific to their responsible area
\req{28} The system must allow agronomists to specify the geographic area in which they are responsible for.
\req{29} The system must allow agronomists to specify the geographic area in which they are responsible for.
\req{30}  The system must allow agronomist to view the list of all farmers in their area.
\req{31}  The system must provide an evaluation of farmers such that the exaluation reflects the quality and quantity of their crop production
\req{32}  The system must enable agronomists to access farmer evaluations from their specific area
\req{33}  The system updates farmers' evaluation when new data is available (ie, new farmer event entries or after an agronomist visit, etc)
\req{33}  The system updates farmers' evaluation when new data is available (ie, new farmer event entries or after an agronomist visit, etc)
\req{34}  The system must provide an interface for daily plans
\req{35}  The system must recommend which farmers should be included in the agronomist's daily plan
\req{36}  The system must generate recommendations such that farmers are visited by their respective agronomists at least twice a year
\req{37}  The system must generate recommendations such that farmers with low evaluation are visited more often than twice a year
\req{38}  The system must allow agronomist to view the list of all farms to visit on a specific day.
\req{39}  The system must allow agronomists to modify which farmers they visit in their plan
\req{40}  The system must allow agronomists to specify and modify the duration of the visits in their plan
\req{41}  The system must maintain a record of farmers who have been visited by their respective agronomists
\end{itemize}

\goal{8} \textbf{Agronomists can specify the deviations from their daily plan and confirm the execution of their daily plan at the end of each day.}
\begin{itemize}
\dom{1}  Users must have a device connected to internet.
\dom{2} To access to the system the user must have valid credentials.
\dom{4} The user has granted permission for GPS, notifications and disk usage
\dom{6} Users can successfully operate an interactive application
\dom{12} Agronomists are assigned an area by their superiors
\dom{13} Agronomists can effectively manage an area assigned to them (ie, the agronomist is not overworked)
\dom{14} Agronomists are experts in their field
\dom{17} Agronomists are effective in determine performance based on various data points
\dom{18} Modifications to the daily plan are simple
\dom{19} If a plan is flagged as confirmed, it has actually been performed by the agronomist


\req{5} The system must provide an interface to visualize data
\req{28} The system must allow agronomists to specify the geographic area in which they are responsible for.
\req{29} The system must allow agronomists to specify the geographic area in which they are responsible for.
\req{30}  The system must allow agronomist to view the list of all farmers in their area.
\req{31}  The system must provide an evaluation of farmers such that the exaluation reflects the quality and quantity of their crop production
\req{32}  The system must enable agronomists to access farmer evaluations from their specific area
\req{33}  The system updates farmers' evaluation when new data is available (ie, new farmer event entries or after an agronomist visit, etc)
\req{35}  The system must recommend which farmers should be included in the agronomist's daily plan
\req{36}  The system must generate recommendations such that farmers are visited by their respective agronomists at least twice a year
\req{37}  The system must generate recommendations such that farmers with low evaluation are visited more often than twice a year
\req{38}  The system must allow agronomist to view the list of all farms to visit on a specific day.
\req{39}  The system must allow agronomists to modify which farmers they visit in their plan
\req{40}  The system must allow agronomists to specify and modify the duration of the visits in their plan
\req{41}  The system must maintain a record of farmers who have been visited by their respective agronomists
\req{42} The system must allow agronomists to modify the daily plan at the end of the day.
\req{43} The system must allow agronomists to confirm that the daily plan was executed that the end of that day
\req{44} The system must not allow anymore modifications to the plan after the plan is confirmed by the agronomist
\req{45} The system must only generate a new plan for a new day after the plan from the preceding day was confirmed by the agronomist.
\end{itemize}

\goal{9} \textbf{Telangana’s policy makers can view the performance of the farmers and the ranking of the farmers.}
\begin{itemize}
\dom{1}  Users must have a device connected to internet.
\dom{2} To access to the system the user must have valid credentials.
\dom{3} The data about weather forecast, the farmers and their production, the sensors, the agronomist are correct, complete and stored by the application
\dom{4} The user has granted permission for GPS, notifications and disk usage
\dom{5} Farmers have an existing system to quantify, track, and organize their production yields
\dom{6} Users can successfully operate an interactive application
\dom{17} Agronomists are effective in determine performance based on various data points


\req{5} The system must provide an interface to visualize data
\req{6} The system must be able to analyze data and show statistics
\req{9} The system must allow the farmer to report production data at a frequency chosen by the farmer.
\req{10} The system must retrieve the weather forecast data from the data that the Telangana government collects.
\req{31}  The system must provide an evaluation of farmers such that the exaluation reflects the quality and quantity of their crop production
\req{33}  The system updates farmers' evaluation when new data is available (ie, new farmer event entries or after an agronomist visit, etc)
\req{46} The system must allow Telangana’s policy makers to view the list of all farmers.
\req{47} The system must allow Telangana’s policy makers to view the performance and evaluation of the farmers.
\req{48} The system must allow Telangana’s policy makers to view the ranking of the farmers.
\req{49} The system must allow Telangana’s policy makers to view well-performing and poor-performing farmers.
\req{50} The system must allow Telangana’s policy makers to flag the farmers that need to be helped based on their performance.
\req{52} The system must allow policy makers to view the history of farmers’ performance/ evaluation (score over time)
\end{itemize}

\goal{10} \textbf{Telangana’s policy makers can use the system to determine if support from agronomists and well-performing farmers produces significant results.}

\begin{itemize}
\dom{1}  Users must have a device connected to internet.
\dom{2} To access to the system the user must have valid credentials.
\dom{3} The data about weather forecast, the farmers and their production, the sensors, the agronomist are correct, complete and stored by the application
\dom{4} The user has granted permission for GPS, notifications and disk usage
\dom{5} Farmers have an existing system to quantify, track, and organize their production yields
\dom{6} Users can successfully operate an interactive application
\req{9} The system must allow the farmer to report production data at a frequency chosen by the farmer.
\dom{12} Agronomists are assigned an area by their superiors
\dom{14} Agronomists are experts in their field
\dom{20} Policy makers want to see the success of farmers in the form of production yields and crop quality.

\req{5} The system must provide an interface to visualize data
\req{6} The system must be able to analyze data and show statistics
\req{9} The system must allow the farmer to report production data at a frequency chosen by the farmer.
\req{10} The system must retrieve the weather forecast data from the data that the Telangana government collects.
\req{28} The system must allow agronomists to specify the geographic area in which they are responsible for.
\req{29} The system must allow agronomists to specify the geographic area in which they are responsible for.
\req{31}  The system must provide an evaluation of farmers such that the exaluation reflects the quality and quantity of their crop production
\req{33}  The system updates farmers' evaluation when new data is available (ie, new farmer event entries or after an agronomist visit, etc)
\req{46} The system must allow Telangana’s policy makers to view the list of all farmers.
\req{47} The system must allow Telangana’s policy makers to view the performance and evaluation of the farmers.
\req{48} The system must allow Telangana’s policy makers to view the ranking of the farmers.
\req{49} The system must allow Telangana’s policy makers to view well-performing and poor-performing farmers.
\req{50} The system must allow Telangana’s policy makers to flag the farmers that need to be helped based on their performance.
\req{51} The system must designate each farmer a measure of support received by agronomists and other well-performing farmers.
\req{52} The system must allow policy makers to view the history of farmers’ performance/ evaluation (score over time)
\req{53} The system must allow Telangana policy makers to view this measure of support designated to each farmer.
\end{itemize}



\end{itemize}
