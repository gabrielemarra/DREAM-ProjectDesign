\subsection{External Interface Requirements}

\subsubsection{User Interfaces}
\subsubsection{App Mockups}

\subsubsection{Hardware Interfaces}
\begin{flushleft}
All users will access the application with their own devices: mobile device, tablet device, or computer device. 
The farmer's device needs the following in order to have full functionality of DREAM:
\begin{itemize}
\item Internet connection [required]: when the farmer is submitting their data while in their fields, their device will need to have an internet connection in order to access DREAM servers.
\item Camera [optional]: farmers can submit pictures to aid their textual descriptions when posting in the discussion board or when messaging their agronomist/s. Although this is not required to use the application, it is highly recommended. 
\item GPS sensor [optional]: farmers can either enter the location for their fields manually or by allowing the application to utilize the GPS sensor from the device. 
\end{itemize}

The agronomist's device needs the following in order to have full functionality of DREAM:
\begin{itemize}
\item Internet connection [required]: when the agronomist is visiting farmers their device will need to have internet access in order for the agronomist to submit their reports to the DREAM servers throughout the day and to follow the navigation path calculated by DREAM.
\item GPS [optional]: since the agronomist will use a navigation plan generated by DREAM, sharing their GPS location is preferred for an optimal user experience. Without this device functionality, the agronomist will not be able to receive turn-by-turn live directions. 
\end{itemize}

The policy maker's device needs the following in order to have full functionality of DREAM:
\begin{itemize}
\item Internet connection [required]: in order to access the rankings and evaluations of the farmers throughout the region, internet connection is required to access the DREAM servers. 
\end{itemize}
\end{flushleft}

\subsubsection{Software Interfaces}
This product is accessible through a mobile-friendly application. Farmers will spend the majority of their time accessing the application through a mobile device; agronomists will access the application with both a mobile device and a computer device; policy makers will access the application almost exclusively with a computer device. Due to the mix of mobile and computer devices, the web application must be suitable for both interfaces. 


\subsubsection{Communication Interfaces}
The mobile-friendly web application will communicate with DREAM servers over an internet connection. 

\subsection{Functional Requirements}
Definition of use case diagrams, use cases and associated sequence/activity diagrams, and mapping on requirements

%Use this counter for the scenarios and command \addOne to print and increase at the same time
\newcounter{usecase_counter}
\setcounter{usecase_counter}{1}

%GENERICS USE CASES
% Use Case Table
\begin{center}
\renewcommand{\arraystretch}{1.25}
\begin{tabular}{|l|>{\raggedright\arraybackslash}m{12cm}|}
    \hline
    \textbf{Name} & User Registration\\
    \hline
   	\textbf{Actor} & Farmer, Agronomist, Policy Maker ???\\
    \hline
    \textbf{Entry Conditions} & \\
    \hline
    
    \textbf{Events Flow} & \begin{enumerate}
    			\item The user register on the site using his/her email.
    			\item The user gets an appointment to verify his/her identity.
    			\item The user profile is validated and obtains credentials.
	    		\end{enumerate}
    	\\
    \hline
    \textbf{Exit Conditions} & \begin{itemize}
    	\item The user has credentials of a validated account.
   		\end{itemize} \\
    \hline
    \textbf{Exceptions} &
    		\begin{itemize}
    			\item 
    		\end{itemize}
    \\
    \hline
\end{tabular}
\end{center}
% Use Case Table

\begin{center}
\begin{tabular}{|l|>{\raggedright\arraybackslash}m{12cm}|}

    \hline
    \textbf{Name} & \textit{User log-in}\\
    \hline
   	\textbf{Actor} & \textit{Farmer, Agronomist, Telangana's policy maker}\\
    \hline
    \textbf{Entry Conditions} & \textit{The user opens the application}\\
    \hline
    \textbf{Events Flow} & \textit{
    \begin{enumerate}
            \item The user clicks the "log-in" option on the screen
            \item The user inserts his/her credentials
            \item The user clicks the "log-in" button
     \end{enumerate}}\\
    \hline
    \textbf{Exit Conditions} & \textit{The system accepts the credentials}\\
    \hline
    \textbf{Exceptions} & \textit{
      \begin{itemize}
          \item The credentials are not valid, therefore the user will be asked to check her/his input
		\item The password is not correct so the user will be asked to insert the correct password. After three tries, the account is temporarily blocked and a reset mail is sent
        \end{itemize}
     }\\
    \hline
\end{tabular}
\end{center}
%FARMERS
\subsubsection{Farmer}
% Scenario Text
\begin{flushleft}
\textbf{Scenario \addOne{usecase_counter}:} 
Max is a farmer who cultivates some fields near his house in Telangana state. During the past years, he planted the same species of plants on his lands because he was scared of trying different ones with unknown needs. Now, with more children and grandchildren to feed, he would like to change the crop with something more productive. 
He heard about the DREAM initiative from a friend and, through the Association of Farmers, he got his credentials to login into the website.
Within a few minutes, he found out in the forums that there are thousands of small farmers like him with the same doubts and fears.
He learned which species are more productive and which fertilizer to use.
Now he can feed his entire family and even sell some food to the local market.
\end{flushleft}
% Use Case Table
\begin{table}[hbt!]
\centering
\caption{\label{tab:addOne{table_counter}}Farmer use case related to Scenario 1.}
\renewcommand{\arraystretch}{1.25}
\begin{tabular}{|l|>{\raggedright\arraybackslash}m{12cm}|}
    \hline
    \textbf{Name} & Create a thread in the discussion forum\\
    \hline
   	\textbf{Actor} & Farmer\\
    \hline
    \textbf{Entry Conditions} & The user logs into the application with valid credentials.\\
    \hline
    \textbf{Events Flow} & 
    		\begin{enumerate}
    			\item The user opens the "Forums" section.
    			\item The user clicks on the "Create Thread" button.
    			\item The user provides a valid title and message.
    			\item The user clicks on the "Publish" button.
    			\item The user can answer messages published in their thread.
    		\end{enumerate}
    	\\
    \hline
    \textbf{Exit Conditions} & \begin{itemize}
    	\item The user exits the "Forums" section.
    	\item The user close the entire application.    
    	\end{itemize}
	\\
    \hline
    \textbf{Exceptions} & 
    		\begin{itemize}
    			\item The server is not available.
    		\end{itemize}
    	\\
    \hline
\end{tabular}
\end{table}
% Use Case Table
\begin{center}
\begin{tabular}{|l|>{\raggedright\arraybackslash}m{12cm}|}

    \hline
    \textbf{Name} & \textit{Visit the farmers forum}\\
    \hline
   	\textbf{Actor} & \textit{Farmer}\\
    \hline
    \textbf{Entry Conditions} & \textit{The user uses valid credentials to log into the application}\\
    \hline
    \textbf{Events Flow} & \textit{
    		\begin{enumerate}
    			\item The user opens the forums section
    			\item The user opens a thread and reads the conversation
    			\item The user answers to an existing thread
    		\end{enumerate}
    	}\\
    \hline
    \textbf{Exit Conditions} & \textit{The user closes the forum section or the entire application}\\
    \hline
    \textbf{Exceptions} & \textit{
    		\begin{itemize}
    			\item The server is not available
    			\item There are no existing thread
    		\end{itemize}
    	}\\
    \hline
\end{tabular}
\end{center}
% Scenario Text
\begin{flushleft}
\textbf{Scenario \addOne{usecase_counter}:} 
Caroline has a big farm in Telangana state with 50 hectares of land and different varieties of plants. Last year, during the monsoon season, her fields were flooded, and almost no plants survived. In addition, this summer, the hot temperature killed some other species.
After consulting an expert, she decided to join the DREAM initiative to ask for direct support from agronomists. Next year she will plant more resilient crops and take some precautions against flooding.
Furthermore, frequently uploading info about her production, she can even monitor the overall performance of her fields and try to obtain some incentives from the central government.
\end{flushleft}
% Use Case Table
\begin{center}
\begin{tabular}{|l|>{\raggedright\arraybackslash}m{12cm}|}

    \hline
    \textbf{Name} & \textit{Asking advice to agronomists}\\
    \hline
   	\textbf{Actor} & \textit{Farmer}\\
    \hline
    \textbf{Entry Conditions} & \textit{The farmer uses valid credentials to login into the application}\\
    \hline
    
    \textbf{Events Flow} & \textit{
    		\begin{enumerate}
    			\item The farmer opens the "Ask to experts" section
    			\item The farmer select the assigned agronomist
    			\item The farmer write a short request and submit it
    			\item After receiving a notification, the farmer can see the answer and, eventually, continue the conversation asking further questions
    		\end{enumerate}
    	}\\
    \hline
    \textbf{Exit Conditions} & \textit{The farmer or the agronomist closes the conversation }\\
    \hline
    \textbf{Exceptions} & \textit{
    		\begin{itemize}
    		    	\item The server is not available
    			\item The agronomist doesn't answer to the message
    		\end{itemize}
    }\\
    \hline
\end{tabular}
\end{center}
% Use Case Table
\begin{center}
\begin{tabular}{|l|>{\raggedright\arraybackslash}m{12cm}|}

    \hline
    \textbf{Name} & \textit{Upload info about production status}\\
    \hline
   	\textbf{Actor} & \textit{Farmer}\\
    \hline
    \textbf{Entry Conditions} & \textit{The farmer uses valid credentials to login into the application and has new data to upload}\\
    \hline
    
    \textbf{Events Flow} & \textit{
    		\begin{enumerate}
    			\item The farmer opens the "My production" section
    			\item The farmer clicks on the "Update production data" button
    			\item The farmer selects the field to update from a list
    			\item The farmer adds the desired info
    			\item The farmer clicks on the "Submit" button
    		\end{enumerate}
    	}\\
    \hline
    \textbf{Exit Conditions} & \textit{The farmer submits the info}\\
    \hline
    \textbf{Exceptions} & \textit{
    		\begin{itemize}
    			\item The server is not available
    			\item The data are not valid
    			\item Some required info are left void
    		\end{itemize}
    }\\
    \hline
\end{tabular}
\end{center}

% Use Case Table
\begin{table}[hbt!]
\centering
\small
\caption{\label{table:farmerNewField}Farmer adding a new field and allowing DREAM to use their GPS location.}
\renewcommand{\arraystretch}{1.25}
\begin{tabular}{|l|>{\raggedright\arraybackslash}m{12cm}|}

    \hline
    \textbf{Name} & Add field location with GPS\\
    \hline
   	\textbf{Actor} & Farmer\\
    \hline
    \textbf{Entry Conditions} & The user logs into the application with valid credentials and is physically in the field.\\
    \hline
    \textbf{Events Flow} & 
    		\begin{enumerate}
    			\item The user opens the "My Farm" section.
    			\item The user clicks on the "Add New Field" button.
    			\item The user is prompted to allow the application to use their GPS location.
    			\item The user clicks on the "Allow" button, allowing the application to locate them using GPS.
    			\item The user adds information about the size of the field, crop currently planted, the fertilizer in use, and any other information they deem relevant.
    			\item The user clicks on the "Submit" button.
    		\end{enumerate}
    	\\
    \hline
    \textbf{Exit Conditions} & The user submits the information about the new field.\\
    \hline
    \textbf{Exceptions} & 
    		\begin{itemize}
    			\item The server is not available.
    			\item The application is not able to use the GPS.
    			\item The user doesn't allow the application to use GPS.
    			\item The provided information is not valid.
    			\item Some required information is left empty.
    		\end{itemize}
    \\
    \hline
\end{tabular}
\end{table}
% Use Case Table
\begin{center}
\begin{tabular}{|l|>{\raggedright\arraybackslash}m{12cm}|}

    \hline
    \textbf{Name} & \textit{See weather forecast}\\
    \hline
   	\textbf{Actor} & \textit{Farmer}\\
    \hline
    \textbf{Entry Conditions} & \textit{The user uses valid credentials to log into the application and has at least one field registered inside the application}\\
    \hline
    
    \textbf{Events Flow} & \textit{
    		\begin{enumerate}
    			\item The user opens the "My fields" section
    			\item The user clicks on the "Weather" tab
    			\item The user selects the field from a list
    			\item The user can see the weather for the selected field
	    		\end{enumerate}
    	}\\
    \hline
    \textbf{Exit Conditions} & \textit{The user closes the weather section or the entire application }\\
    \hline
    \textbf{Exceptions} & \textit{
    		\begin{itemize}
    			\item The server is not available
    		\end{itemize}
    }\\
    \hline
\end{tabular}
\end{center}

\newpage
%AGRONOMISTS
\subsubsection{Agronomist}
%before: emails for help, data, now can consult

% Scenario Text
\begin{flushleft}
\textbf{Scenario \addOne{usecase_counter}:} 
Schmidt is an agronomist. His job is based on knowing farmers' information and helping them to improve and solve problems. Before installing DREAM he was receiving various emails from farmers. All of them contained information about them and help or suggestion requests. He then needed to combined them with the information he provided about latest visits of their farm. Sometimes data were not in the same format and not standardized this slowed so much his work. Now with DREAM he can easily inspect all of the data in the same place, his device. Data are more readable and available. He can also now consult them in any place with an internet connection.
\end{flushleft}

%respond to help req
% Use Case Table

\begin{center}
\renewcommand{\arraystretch}{1.25}
\begin{tabular}{|l|>{\raggedright\arraybackslash}m{12cm}|}

    \hline
    \textbf{Name} & Respond to help requests\\
    \hline
   	\textbf{Actor} & Agronomist\\
    \hline
    \textbf{Entry Conditions} & The user has opened the application, logged in with valid credentials, specified their responsible area, and received at least one help request.\\
    \hline
    \textbf{Events Flow} & \begin{enumerate}
            \item The user clicks on the "Help Requests" button.
            \item The user views the list of the the different help requests received from different farmers.
            \item The user can select a farmer from the list of help request and inspect data about the farmer such as location, ranking, and evaluation.
            \item The user reads the help requests and can then reply with solutions and suggestions.
       \end{enumerate}\\
    \hline
    \textbf{Exit Conditions} & \begin{itemize}
    	\item The user has answered to the farmer and the response has been sent to the farmer.
    	\item The user navigates out of the "Help Requests" section.
    \end{itemize}\\
    \hline
    \textbf{Exceptions} & 
       \begin{itemize}
          \item The messages are not accessible at the moment the user is asked to try again later.
          \item A message cannot be sent in this moment the user is asked to try again later.
        \end{itemize}
     \\
    \hline
\end{tabular}
\end{center}

%visits, daily plan, info on performance, suggests path, confirm plan
% Scenario Text
\begin{flushleft}
\textbf{Scenario \addOne{usecase_counter}:} 
Cece is an agronomist. Part of her job is to visit farms in the area. She has to establish a daily plan for her visits and then send all the data she collects to the Telangana authorities. The daily plan is very hard to create: she has to search for farmers that need more help than others and find an itinerary that connects all their locations. She has to mind the distance between farms and how to reach them. She installs DREAM and now her work is much easier. All of the information about location, performance, and needs is in one place. The system also recommends farmers to visit and automatically suggests a path connecting all the farms. At the end of a day of visits, she has to compile the forms about farmers' performance that are sent automatically to the Telangana authorities.
\end{flushleft}

%create daily plan
% Use Case Table

\begin{center}
\begin{tabular}{|l|>{\raggedright\arraybackslash}m{12cm}|}

    \hline
    \textbf{Name} & \textit{Daily plan creation}\\
    \hline
   	\textbf{Actor} & \textit{Agronomist}\\
    \hline
    \textbf{Entry Conditions} & \textit{The agronomist has opened the app and is logged in}\\
    \hline
    \textbf{Events Flow} & \textit{\begin{enumerate}
            \item The user clicks on the daily plans interface
            \item The user clicks on create a new daily plan
            \item The user chooses a day from a calendar
            \item The user can then choose multiple farmers to visit that day from a list of recommended ones
            \item Whenever a farmer is added to the daily plan the system recalculates the best path that connects all selected farmers and updates the map
       \end{enumerate}}\\
    \hline
    \textbf{Exit Conditions} & \textit{The daily plan is created and inserted in the “your daily plans” section by the system}\\
    \hline
    \textbf{Exceptions} & \textit{
      \begin{itemize}
          \item The server is not responding so the user is asked to try again later
          \item The google API is not working at the moment so the user is asked to try again later
        \end{itemize}
     }\\
    \hline
\end{tabular}
\end{center}
%confirm daily plan
% Use Case Table

\begin{center}
\begin{tabular}{|l|>{\raggedright\arraybackslash}m{12cm}|}

    \hline
    \textbf{Name} & \textit{Confirmation of the daily plan}\\
    \hline
   	\textbf{Actor} & \textit{Agronomist}\\
    \hline
    \textbf{Entry Conditions} & \textit{The user has opened the app, is logged in and has inserted the area is responsible of and has a daily plan created for that specific day}\\
    \hline
    \textbf{Events Flow} & \textit{\begin{enumerate}
            \item The user click on the daily plan interface
            \item The user clicks on the "your daily plans" 
            \item The user can then choose a specific daily plan and view information about it or modify it
       \end{enumerate}}\\
    \hline
    \textbf{Exit Conditions} & \textit{The plans are correctly modified and saved by the system}\\
    \hline
    \textbf{Exceptions} & \textit{
       \begin{itemize}
          \item The database with daily plans data is not accessible at the moment the user is asked to try again later
        \end{itemize}
     }\\
    \hline
\end{tabular}
\end{center}

%viewing ranking
% Use Case Table

\begin{table}[hbt!]
\centering
\small
\caption{\label{tab:agrViewRank}Agronomist Use Case: View Rank.}

\begin{tabular}{|l|>{\raggedright\arraybackslash}m{12cm}|}

    \hline
    \textbf{Name} & Viewing the ranking\\
    \hline
   	\textbf{Actor} & Agronomist\\
    \hline
    
    %The user logs into the application with valid credentials
    \textbf{Entry Conditions} & The user has opened the application, logged in with valid credentials, and specified their responsible area. \\
    \hline
    \textbf{Events Flow} & \begin{enumerate}
            \item The user clicks on the farmers ranking interface.
            \item The user can view general statistics of the area.
            \item The user can click on any farmer to view their evaluation and statistics.
       \end{enumerate}\\
    \hline
    \textbf{Exit Conditions} & The user selects another tab to exit the ranking view.\\
    \hline
    \textbf{Exceptions} & 
       \begin{itemize}
          \item The database of farmers performance is not accessible.% at the moment the user is asked to try again later.
        \end{itemize}
     \\
    \hline
\end{tabular}
\end{table}
% sending report
% Use Case Table
\begin{table}[hbt!]
\centering
\caption{\label{tab:addOne{figure_counter}}Use case for Agronomist.}

\renewcommand{\arraystretch}{1.25}
\begin{tabular}{|l|>{\raggedright\arraybackslash}m{12cm}|}
    \hline
    \textbf{Name} & Sending a report\\
    \hline
   	\textbf{Actor} & Agronomist\\
    \hline
    \textbf{Entry Conditions} & The user has opened the application, logged in with valid credentials, and specified their responsible area.\\    
    \hline
    \textbf{Events Flow} & 
    	\begin{enumerate}
            \item The user clicks on the daily plan interface.
            \item The user click on the daily plan of the day and selects a farmer also based on the GPS position.
            \item The user can fill a form about the farmer's data, evaluation and 
       \end{enumerate}\\
    \hline
    \textbf{Exit Conditions} & The user has correctly sent the information to the server and the data have been saved.\\
    \hline
    \textbf{Exceptions} & 
    	\begin{itemize}
	    	\item The server is not accessible at the moment.
    	\end{itemize}\\
    \hline 
\end{tabular}
\end{table}


%choosing location
% Use Case Table

\begin{center}
\begin{tabular}{|l|>{\raggedright\arraybackslash}m{12cm}|}

    \hline
    \textbf{Name} & \textit{Choosing the location}\\
    \hline
   	\textbf{Actor} & \textit{Agronomist}\\
    \hline
    \textbf{Entry Conditions} & \textit{The user has opened the app and is logged in}\\
    \hline
    \textbf{Events Flow} & \textit{\begin{enumerate}
            \item The user open the "location" section of the application
            \item The user selects the area he/she his responsible of on a map
       \end{enumerate}}\\
    \hline
    \textbf{Exit Conditions} & \textit{The system saves the area chosen and the user is now able to perform actions inside the application }\\
    \hline
    \textbf{Exceptions} & \textit{
        \begin{itemize}
          \item The site is not accessible at the moment so the user is asked to try again later
        \end{itemize}
     }\\
    \hline
\end{tabular}
\end{center}

%inspect weather forecast
% Use Case Table
\begin{center}
\renewcommand{\arraystretch}{1.25}
\begin{tabular}{|l|>{\raggedright\arraybackslash}m{12cm}|}
    \hline
    \textbf{Name} & Inspecting weather forecasts\\
    \hline
   	\textbf{Actor} & Agronomist\\
    \hline
    \textbf{Entry Conditions} & The user has opened the application, logged in with valid credentials, and specified their responsible area.\\    
    \hline
    \textbf{Events Flow} & 
    	\begin{enumerate}
            \item The user clicks on the weather forecast interface.
            \item The user clicks on a section of their responsible area.
            \item The user inspects the weather forecast for the selected area.
       \end{enumerate}\\
    \hline
    \textbf{Exit Conditions} & The user exits the weather forecast section.\\
    \hline
    \textbf{Exceptions} & 
    	\begin{itemize}
	    	\item The data about weather forecast is not accessible.
    	\end{itemize}\\
    \hline 
\end{tabular}
\end{center}


\newpage
%POLICY MAKERS
\subsubsection{Policy Maker}
%\begin{flushleft}
\textbf{Scenario :} 
Ernie is a policy maker within the Teleagana government tasked with identifying the poor-performing farmers in the region. Ernie is a liaison between the Telangana government and local suppliers for farming supplies so he is aware that this year a popular fertilizer manufacturer called NaturGrow slightly changed their formula. Ernie logs in as a policy maker and navigates to the main ranking view of all the farmers in the region. Ernie configures his view of production scores produced by DREAM to compare by type of fertilizer used. Ernie cannot see a discernible difference between the different fertilizers. Ernie then compares this view to scores from the last 5 seasons, again comparing the type of fertilizer used, and Ernie notices that farmers using the new formula of NaturGrow all increased their yields by at least 8\% whereas other farmers that did not use NaturGrow saw a change in their yields by -2 to 3\%. Ernie uses this information to flag the farmers who observed a reduction in their yields and the farmers who observed an increase in their yields of less than 8\%. In total Ernie flags 24 farmers. 
\end{flushleft}


\begin{flushleft}
\textbf{Scenario \addOne{usecase_counter}: Setting Triggers}\\\smallskip
% manually flagged outliers, set trigger to flag future dilenquient farmers
Winston is a policy maker within the Telangana government tasked with providing a mid-season status report on the progress of the farmers. Winston configures his view to organize the ranking of the farmers by water usage. Winston notices a group of outliers over-consuming water but producing below-average yields. Winston sets a trigger flag farmers that surpass the average water usage while yielding below-average production. The flagged farmers have increased priority in the queue for recommended farmers to visit on the agronomist interface. 
\end{flushleft}


% setting triggers

% Use Case Table
\begin{table}
\centering
\caption{\label{tab:addOne{table_counter}}taptaptap.}
\renewcommand{\arraystretch}{1.25}
\begin{tabular}{|l|>{\raggedright\arraybackslash}m{12cm}|}

    \hline
    \textbf{Name} & \textit{Set Trigger}\\
    \hline
   	\textbf{Actor} & Policy Maker\\
    \hline
    \textbf{Entry Conditions} & The user is logged in with valid credentials.\\
    \hline
    \textbf{Events Flow} & 
    \begin{enumerate}
	    \item The user clicks on "Create Trigger" button.
    	\item The user is navigated to the "Create Trigger" wizard.
    	\item The user selects the parameters and limits for the trigger.
    	\item The user clicks on the "Submit" button.
    \end{enumerate} \\ \hline
    \textbf{Exit Conditions} & Sufficient information is provided to set the trigger and the trigger is entered into the system.\\
    \hline
    \textbf{Exceptions} & \begin{itemize}
    	\item Limits are missing for the parameters selected.
    	\item The database is unavailable.
    \end{itemize}\\
    \hline
\end{tabular}
\end{table}

%flagging poor perf

% Use Case Table
\begin{center}
\begin{tabular}{|l|>{\raggedright\arraybackslash}m{12cm}|}

    \hline
    \textbf{Name} & Flag Farmers\\
    \hline
   	\textbf{Actor} & Policy Maker\\
    \hline
    \textbf{Entry Conditions} & User is logged in as a policy maker with valid credentials. User has configured a ranking view.\\
    \hline
    \textbf{Events Flow} & \begin{enumerate}
    \item Policy maker chooses to rank farmers at one extremity of the ranking list. 
    \item Policy maker scrolls to the desired extremity of the list.
    \item Policy maker clicks on the grey "flag" icon on the top right of the tile for each respective farmer they would like to flag. 
    \end{enumerate}\\
    \hline
    \textbf{Exit Conditions} & The flag icon turns red \\
    \hline
    \textbf{Exceptions} & None\\
    \hline
\end{tabular}
\end{center}

\begin{flushleft}
\textbf{Scenario \addOne{usecase_counter}: DREAM helping drive policy decisions}\\\smallskip
The Telangana government partners with a start-up that is creating a new agriculture tool aimed at reducing the manual labor involved in cultivating castor. Jessica, a Telengana policy maker, rolls out a beta program to issue this new product to various farmers across the region to observe how the product affects castor crop quality and production yields. In the beginning of the program, Jessica creates a watch-list consisting of the farmers involved in the beta program. Mid-way through the season, Jessica observes that farmers participating in the beta program are generating much greater production yields compared to farmers not in the beta program. The increased production yields qualifies the program to get rolled out to the rest of the region. Jess uses the percent-increase data to determine how much the Telangana government can subsidize the cost of the tool. 
\end{flushleft}


%configure and view ranking
% Use Case Table
\begin{center}
\begin{tabular}{|l|>{\raggedright\arraybackslash}m{12cm}|}

    \hline
    \textbf{Name} & \textit{Flag Farmers}\\
    \hline
   	\textbf{Actor} & Policy Maker\\
    \hline
    \textbf{Entry Conditions} & \textit{Enter the entry conditions required for this use case to be relevant/ applicable}\\
    \hline
    \textbf{Events Flow} & \textit{Enter the flow flow splash splash of events}\\
    \hline
    \textbf{Exit Conditions} & \textit{Enter the circumstances required to exit this use case situation}\\
    \hline
    \textbf{Exceptions} & \textit{Enter and exceptions}\\
    \hline
\end{tabular}
\end{center}


\begin{figure}[hbt!]
\centering
\includegraphics[scale=0.6]{../images_diagrams/usecasediagram.png}
\caption{\label{fig:usecase}Use case diagram.}
\end{figure}

\newpage
\subsection{Sequence Diagrams}

\subsubsection{Farmer}
\begin{figure}[hbt!]
\centering
\includegraphics[scale=0.6]{Files/sequence_disgrams/thePNGs/farmer_askExperts.png}
\caption{\label{fig:farmerSeqRequest}Sequence Diagram for Farmer.}
\end{figure}

\begin{flushleft}
As demonstrated in the sequence diagram, when the farmer user access the "Ask Experts" area in the application, the farmer will send a request for help to their assigned agronomist in the form of a message. The agronomist and the farmer can continue to exchange messages until either user chooses to end the conversation.
\end{flushleft}



\begin{figure}[hbt!]
\centering
\includegraphics[scale=0.6]{Files/sequence_disgrams/thePNGs/farmer_createThread.png}\\
\caption{\label{tab:farmerSeqNewThread}Sequence Diagram for Farmer.}
\end{figure}

\begin{flushleft}
This sequence diagram shows the exchange of information between the user and the DREAM application when the farmer user creates a new thread. After the user publishes the new thread, the DREAM application has to save the submitted form and update its servers.
\end{flushleft}


\begin{figure}[hbt!]
\centering
\includegraphics[scale=0.6]{Files/sequence_disgrams/thePNGs/farmer_newField.png}\\
\caption{\label{tab:farmerSeqNewField}Sequence Diagram for Farmer.}
\end{figure}

\begin{flushleft}
This sequence diagram shows how the farmer user interacts with the DREAMs application when adding a new field to their farm. Part of adding a new field to their farm involves providing the GPS location (if the user chooses to consent the DREAM application to access their location) and entering some information about the field such as the crop planted and the fertilizer used. After submitting the necessary information, the DREAM application saves and updated its servers and the farmer can then view the new field in their "My Fields" interface. 
\end{flushleft}



\subsubsection{Agronomist}

\begin{figure}[hbt!]
\centering
\includegraphics[scale=0.6]{Files/sequence_disgrams/thePNGs/agronomist_choosingLocation.png}\\
\caption{\label{fig:agrSeqArea}Sequence Diagram for Agronomist.}
\end{figure}

\begin{flushleft}
This sequence diagram shows how the agronomist user interacts with the DREAM servers when modifying the area that they are responsible of with a map display interface. 
\end{flushleft}


\begin{figure}[hbt!]
\centering
\includegraphics[scale=0.6]{Files/sequence_disgrams/thePNGs/agronomist_createPlan.png}\\
\caption{\label{fig:agrSeqCreatePlan}Sequence Diagram for Agronomist.}
\end{figure}
\begin{flushleft}
This sequence diagram shows how the agronomist user interacts with the DREAM servers when creating a new daily plan in the daily plan interface including some of the specific calls that occur between the DREAM servers and the user's device. As the agronomist modifying the list of farmers to visit for the day, the agronomist's map is updated to show the new navigation path. From the list of farmers to visit, the agronomist can confirm or cancel /remove each farmer from the plan. Then, at the end, the agronomist can confirm the daily plan and the data is updated on the DREAM servers. 
\end{flushleft}

\begin{figure}[hbt!]
\centering
\includegraphics[scale=0.6]{Files/sequence_disgrams/thePNGs/agronomist_confirmPlan.png}\\
\caption{\label{fig:agrSeqConfirmPlan}Sequence Diagram for Agronomist.}
\end{figure}

\begin{flushleft}
This sequence diagram shows how the agronomist user interacts with the DREAM servers when confirming a daily plan. Confirmation of a daily plan should occur after the agronomist has completed all the farmers they intend to visit that day. When confirming, the agronomist can modify the plan so that the plan accurately reflects the visits that the agronomist actually completed that day.
\end{flushleft}


\begin{figure}[hbt!]
\centering
\includegraphics[scale=0.6]{Files/sequence_disgrams/thePNGs/agronomist_sendReport.png}\\
\caption{\label{fig:agrSeqSendReport}Sequence Diagram for Agronomist.}
\end{figure}

\subsubsection{Policy Maker}



\begin{figure}[hbt!]
\centering
\includegraphics[scale=0.6]{Files/sequence_disgrams/thePNGs/policy_setFlag.png}\\
\caption{\label{fig:policySeqSetFlag}Sequence Diagram for Policy Maker.}
\end{figure}

\begin{figure}[hbt!]
\centering
\includegraphics[scale=0.6]{Files/sequence_disgrams/thePNGs/policy_setTrigger.png}\\
\caption{\label{fig:policySeqSetTrig}Sequence Diagram for Policy Maker.}
\end{figure}

\begin{figure}[hbt!]
\centering
\includegraphics[scale=0.6]{Files/sequence_disgrams/thePNGs/policy_viewRanking.png}\\
\caption{\label{fig:policySeqViewRank}Sequence Diagram for Policy Maker.}
\end{figure}


\subsection{Performance Requirements}
- many users
- expected widespread use among farmer population
-policy makers are not many, about 100 users
- main thing will be the amount of data to handle
- 


\subsection{Design Constraints}




\subsubsection{Standards compliance}
-security?
-geolocation?
-government 
\subsubsection{Hardware limitations}
- for the policy makers, thier screen must be sufficient to generate good and accurate visutalizations of the data
- gps sensors on devices are not required because users can enter location data manually but it would 1) make for a more seamless user expereince and 2) more convenient for the user
- cameras are not required but definitely can be helpful and effective when farmers communicate their issues in their requests for help
- need good internet connects so devices must be somewhat current 
- 

\subsubsection{Any other constraints}
since we are going to be working with alot of data, the interface should be simple enough as to not overwhelm the user but also ensure that the user has access to all the feature, tools, and configurations (ie, policy makers and agronomists configuring ranking views, or farmers providing qualitative data. 



\subsubsection{Reliability}
- data backups 


mtbf: mean time between failure
mttf: mean time to failure
mttr: mean time to recover
\subsubsection{Availability}
policy makers will work exclusively during buisness hours: access to DREAM servers msut be available
agronomists will also work during business hours so this is also just dream servers
farmers may have extended hours but suggestions (access to external db generated by agronomists can be made; also a possibility: farmers can complete their data entry while physically in their fields but then the app submits their data when they get stable internet connection

also access to a db can be done during evenings. off hours when traffic across the DREAM servers is low to distribute strain on the system
\subsubsection{Security}

log in credentials may be extra sensitive because some users are government employees
farmer data is senstive, because 1) their livelihood and 2) address information
\subsubsection{Maintainability}
scalability as data collection grows
good coding practices
adherence to standards

\subsubsection{Portability}
android devices
ios devices
web browser

