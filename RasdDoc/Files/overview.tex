Here you can see how to include an image in your document.

\subsection{Product Perspective}
\begin{flushleft}

Here we include scenarios and further details on the shared phenomena and a domain model (class diagrams and statecharts)
\end{flushleft}

\begin{itemize}
\item
include a class diagram?
\item
flow chart of system?
\item
main objective of this system is to serve as an interface (intermediary?) between the farmers, agronomists, and policy makers. 
\item
intention is that this system will incorporate data-driven [blank] and include community focused [blank]? 
\item
this system will bridge the gap across these stakeholders with the goal for the telengana to to produce more effective and data-driven policies.
\item
since this is a big problem, with many involved people, people geographically spread out, etc this system provides an interface to that data can be shared and effectively used across users. 
\item
strengthen the cohesiveness of the policies generated and informed by data provided by users
\end{itemize}


\subsection{Product Functions}
\begin{flushleft}

Below is a description of all the main functions of the product. The functions described are generalizations of the specific goals of the product. There functions are further specified in the requirements section and are somewhat (loosely) demonstrated in the scenarios section. 

\subsubsection{Data Visualization}
DREAMS helps the farmers to visualize and analyze data about their production: providing simple graphs and statistics to the users, the app allows them to better understand their needs.In the section named "My Fields" each user can find personalized suggestions: thanks to the data provided by the farmers, DREAMS can recommend the type of crop to plant or the best fertilizer to use.DREAMS also provides the users a precise weather forecast that is constantly updated.

\subsubsection{Support for Farmers and Discussion Forums}
With DREAMS, each farmer can easily ask for help or suggestions directly from professional agronomists: inside the "Ask experts" section, the user can send messages to agronomists and obtain vital information to improve their production.Another opportunity is the "Farmers Forum" where each farmer can ask for help from other farmers or read answers to already posted questions.

\subsubsection{Performance Visualization and Evaluations}
The DREAM system offers agronomists and policy makers a visualization of the performance of the farmers by means of a ranked list view. Agronomists only visualize the performance of the farmers inside their responsible area whereas policy makers can visualize the performance of all the farmers involved in the DREAM initiative. By default, this visualization of the ranking organizes farmers by their overall score which combines data such as crop quality, production yields, water usage, and evaluations by the agronomists. Users can also configure their ranking view to filter by one of these metrics, by location, or by other attributes of the farmers. \\
\smallskip
In addition to a visualization of performance, policy makers can also access evaluations of the farmers issued by agronomists that enable the policy makers to determine the efficacy of the support provided to the farmer. 

\subsubsection{Daily Plan}
The DREAM system provides the agronomists a tool for managing their daily plans with functions such as plan creation, modification and confirmation. During the creation of the daily plan the system recommends farmers to visit based on the performance of the farmers and the \hl{and frequency of visits of their farms}. The system also displays a map containing all the fields of the farmers. As the agronomist modifies their list of farmers to visit as part of the daily plan, the DREAM system updates the travel path connecting the fields. During the confirmation of the daily plan, the system allows the agronomist insert all the data about the visits they performed during the day.
\end{flushleft}

\subsection{User Characteristics}
\begin{flushleft}
Users of the system fall into one of the following categories: farmer, agronomist, or policy maker.
\subsubsection{Farmer}
A farmer is the type of user that intends to use the DREAM system to visualize data relevant to them such as: weather forecasts or personalized suggestions. Personalized suggestions generated by the DREAM product may consist of new crops to consider planting, different fertilizers to try, or other cultivating methods to attempt. Farmers can also use the DREAM system to ask professional agronomists and other nearby farmers for help and suggestions.\\
\subsubsection{Agronomist}
An Agronomist is a type of user that intends to use the DREAM system as a tool to manage some of their job responsibilities. Agronomists want to: receive requests for help from the farmers; answer requests from farmers; inspect rankings of the performance of farmers; provide evaluations regarding farmers performance; utilize a tool to create, modify and confirm daily plans to manage farms visits.
\subsubsection{Policy Maker}
A Telangana policy maker is a type of user that intends to use the DREAM system to drive policy decisions. Policy makers are mainly interested in accessing rankings and evaluations of all the farmers in the entire area as well as identifying broader trends in the data such as relating the community-provided support to production outcomes. Since policy makers have a more holistic view of the region, they use the DREAM system to configure metrics that are used to classify "well-performing" and "poor-performing" farmers based on rankings, evaluations, and data.\\
\end{flushleft}

\subsection{Assumptions, dependencies and constraints}
\begin{flushleft}
The following table lists the domain assumptions relevant to the context in which the DREAM system operates.
\end{flushleft} 

% Assumptions Table
\newcounter{assum_counter}
\setcounter{assum_counter}{1}

\begin{table}
\centering
\caption{\label{tab:addOne{table_counter}}Domain Assumptions.}

\renewcommand{\arraystretch}{1.25}
\begin{tabular}{|l|>{\raggedright\arraybackslash}m{12cm}|} \hline
    \textbf{ID} & \textbf{Domain Assumption}\\\hline
	D\addOne{assum_counter} & Users must have a device connected to internet.\\\hline
	D\addOne{assum_counter} & To access to the system the user must have valid credentials.\\\hline
	D\addOne{assum_counter} & The data about weather forecast, the farmers and their production, the sensors, the agronomist are correct, complete and sent to the application. \\\hline
	D\addOne{assum_counter} & The user has granted permission for GPS, notifications and disk usage.\\\hline
	D\addOne{assum_counter} & Farmers have an existing system to quantify, track, and organize their production yields.\\\hline
	D\addOne{assum_counter} & Users can successfully operate an interactive application.\\\hline
	D\addOne{assum_counter} & Business competition will not influence the farmers' willingness to help.\\\hline
	D\addOne{assum_counter} & Farmers are willing to ask for help from other farmers and/or agronomists.\\\hline
	D\addOne{assum_counter} & Farmers have industry knowledge about fertilizers, crops, etc.\\\hline
	D\addOne{assum_counter} & Farmers are willing to interact with other farmers.\\\hline
	D\addOne{assum_counter} & Farmers can recognize issues and production abnormalities.\\\hline
	D\addOne{assum_counter} & Agronomists are assigned an area by their superiors.\\\hline
	D\addOne{assum_counter} & Agronomists can effectively manage an area assigned to them (ie, the agronomist is not overworked).\\\hline
	D\addOne{assum_counter} & Agronomists are experts in their field.\\\hline
	D\addOne{assum_counter} & Agronomists will be effective in addressing issues farmers face.\\\hline
	D\addOne{assum_counter} & Agronomists have access to an internet connection.\\\hline
	D\addOne{assum_counter} & Agronomists can successfully operate an interactive application.\\\hline
	D\addOne{assum_counter} & Weather forecast data is available.\\\hline
	D\addOne{assum_counter} & Weather forecast data is accurate.\\\hline
	D\addOne{assum_counter} & Farmers are not interested in meteorological changes that occur in less than 5 minutes.\\\hline
	D\addOne{assum_counter} & Agronomists are effective in determine performance based on various data points.\\\hline
	D\addOne{assum_counter} & Modifications to the daily plan are simple.\\\hline
	D\addOne{assum_counter} & confirmed plans actually happened /..... [better worded].\\\hline
	D\addOne{assum_counter} & Policy makers want to see the success of farmers in the form of production yields and crop quality.\\\hline
\end{tabular}
\end{table}

