\begin{flushleft}
\textbf{Scenario 3: DREAM helping drive policy decisions} 
Castor is becoming increasingly popular in the cosmetics and machining oils markets. The Telangana government is partners with a start-up that is creating a new agriculture tool aimed at reducing the manual labor involved in cultivating castor. Jessica, a Telengana policy maker, rolls out a beta program to issue this new product to various farmers across the region to observe how the product affects castor crop quality and production yields. In the beginning of the program, Jessica logs into the DREAM system as a policy maker and creates a watch-list from the farmers involved in the beta program. Mid-way through the season, Jessica observes that farmers participating in the beta program are generating 20\% higher castor yields but comparable crop quality scores compared to farmers not in the beta program. The increased production yields qualifies the program to get rolled out to the rest of the region. Jess uses the percent-increase data to determine how much the Telangana government can subsidize the cost of the tool. 
\end{flushleft}
